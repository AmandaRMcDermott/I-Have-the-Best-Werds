\documentclass[12pt]{article}
% === graphic packages ===
%\usepackage{epsf,graphicx,psfrag}
\usepackage{graphicx}
\usepackage{lscape}
% === bibliography package ===
\usepackage{natbib}
% === margin and formatting ===
\usepackage[top=1in, bottom=1in, right=1in, left=1in]{geometry}
\usepackage{setspace}
%\usepackage{vmargin}
%\setpapersize{USletter}
% === math packages ===
\usepackage[reqno]{amsmath}
\usepackage{amssymb}
% === dcolumn package ===
\usepackage{dcolumn}
\newcolumntype{.}{D{.}{.}{-1}}
\newcolumntype{d}[1]{D{.}{.}{#1}}
% === additional packages ===
\usepackage{url}
% === definitions ===
\newcommand{\bibtex}{\textsc{Bib\TeX}}
% === title, author, etc. ===
\title{%
  Talking Heads of State \
  \large Text Based Analysis of Executive Speeches to Domestic and International Audiences}
\author{Austin Jang and Amanda McDermott}
\date{December 2018}
\begin{document}
\maketitle
% === abstract ===
\begin{abstract}
We look at speeches from a wide variety of countries inclduing: China, the Philippines, Russia, South Africa, and the United States. We analyze "state of the union" speeches.
We analyze speeches given at the United Nation general debates.
\end{abstract}
% === spacing ===
\doublespacing
\tableofcontents
\clearpage
%=== text starts here ===
\section{Introduction}
\label{sec:intro}

While the US State of the Union address if obviously not conducted in any other country besides the US, there are
foreign equivalents to the state of the union. For example, in the Philippines, the president delivers an annual 'state
of the nation' address. In China, the president delivers a report to the CPC every five years. While these speeches may vary
on certain elements ubiquitous to each country, we would contend that these speeches do share certain features that make them
comparable. Namely, first, the content of the speeches are determined by the executive, second, the speeches are delivered at
regular intervals, and third, the speeches are directed at their respective legislative branches, but are also heard by the people of the
state at large.

\section{Data Collection}
\label{sec:data}

All text data for the UN general debates was provided by \cite{baturo:08}. \cite{baturo:08} provided cleaned .txt files of all UNGD
speeches given from 1970-2018.  On the other hand, text data for SOTU speeches was much more difficult to find, as expected,
different countries SOTU speeches were not all located in one central location. 

\section{Conclusion}
\label{sec:conc}

Presumably, we can't go on forever and will need to conclude at some point.

\section{Appendix A: Locating State of the Union Speeches}
\label{sec:appa}

The United States' State of the Union speeches were scraped from xxxx (Amanda, wanna fill me in?), which keeps an archive of all state of the union speeches.

The Philippines' State of the Nation were scraped from the Official Gazette, which keeps an archive of all state of the nation speeches up until, but not including, Rodrigo Duterte.
https://www.officialgazette.gov.ph/past-sona-speeches/

China's CCP Congress reports were manually scraped from the China media project, up until, but not including the 19th CCP report delivered by Xi Jinping in 2017. This was manually scraped from China Daily.

http://chinamediaproject.org/2017/10/05/six-decades-of-ccp-congress-reports/
http://www.chinadaily.com.cn/china/19thcpcnationalcongress/2017-11/04/content\textunderscore34115212.htm

South Africa's State of the Nation addresses were scraped from South Africa's government website which provides copies of every address since independence.
https://www.gov.za/state-nation-address 

\singlespacing 
\bibliographystyle{apsr}
\bibliography{paper}

\end{document}


